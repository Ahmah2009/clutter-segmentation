\clearemptydoublepage

\phantomsection
\addcontentsline{toc}{chapter}{Outline}

\begin{center}
	\huge{Outline}
\end{center}

%--------------------------------------------------------------------
\section*{Part I: Introduction}

\noindent {\scshape Chapter 1: Statement of the Problem}  \vspace{1mm}

The reader learns about the outcome of the work, and about the contribution of
this work to the field. The reader learns about existing other work in this
area.

\noindent {\scshape Chapter 2: Intended Purpose}  \vspace{1mm}

The reader learns why this work has been done and where the system can be used.

\noindent {\scshape Chapter 3: Selected Approach}  \vspace{1mm}

The reader learns about the methods that have been employed.

%--------------------------------------------------------------------
\section*{Part II: Theory}

\noindent {\scshape Chapter 1: Object Recognition}  \vspace{1mm}

The reader learns about the terminology used in object recognition,
especially with regard to this paper. The reader learns about existing work in
object recognition that is important for this work. The reader learns about
different approaches to object recognition.

\noindent {\scshape Chapter 2: Classifier Evaluation}  \vspace{1mm}

The reader learns about the terminology used in classifier evaluation,
especially with regard to this paper. The reader learns about how to view
the object recognition task as a classification problem. 

%--------------------------------------------------------------------
\section*{Part III: Implementation}

\noindent {\scshape Chapter 1: Model Learning}  \vspace{1mm}

The reader learns about how models are generated that lay the foundation for
later classification. The reader learns how certain difficulties have been
resolved. The reader learns about the learning parameters and their effects.

\noindent {\scshape Chapter 2: Object Recognition}  \vspace{1mm}

The reader learns how the classification process works and how the pose of the
object is estimated. The reader learns about the design choices.  The reader
learns about the parameters that can be adjusted to get different classifiers.

\noindent {\scshape Chapter 3: Parameter Selection}  \vspace{1mm}

%--------------------------------------------------------------------
\section*{Part IV: Results}

\noindent {\scshape Chapter 1: Performance in Experiments}  \vspace{1mm}

The reader learns about performance on datasets. The reader learns about how much
influence 3d information in models has on classification. The reader learns about
how long classification takes. 

\noindent {\scshape Chapter 2: Future Work}  \vspace{1mm}

The reader learns which areas have not been sufficiently investigated into,
which issues remain and where to improve.

\noindent {\scshape Chapter 3: Conclusion}  \vspace{1mm}

The reader is reminded what this work was about. The reader learns about what is
different to other works and possible usage scenarios are presented.
% 
% %--------------------------------------------------------------------
% \section*{Part I: Introduction and Theory}
% 
% \noindent {\scshape Chapter 1: Introduction}  \vspace{1mm}
% 
% \noindent  This chapter presents an overview of the thesis and it purpose. Furthermore, it will discuss the sense of life in a very general approach.  \\
% 
% \noindent {\scshape Chapter 2: Theory}  \vspace{1mm}
% 
% \noindent  No thesis without theory.   \\
% 
% %--------------------------------------------------------------------
% \section*{Part II: The Real Work}
% 
% \noindent {\scshape Chapter 3: Overview}  \vspace{1mm}
% 
% \noindent  This chapter presents the requirements for the process.
